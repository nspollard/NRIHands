\section{Introduction}

Robot dexterity is critically important for robots working alongside people and for robots working in human spaces.    Robots must be able to manipulate human objects and use tools made for people in order to assist people in performing tasks from mission critical to everyday.   

Unfortunately, despite decades of development of high degree-of-freedom dexterous robot hands since the 80's, dexterous manipulation has remained elusive.  Dexterity is a Grand Challenge goal.   In fact, the 2013 Robotics Roadmap [REFERENCE THIS PROPERLY] states:
\begin{quotation}
{\small \it
Robot arms and hands will eventually out-perform human hands. This is already true in terms of speed and strength. However, human hands still out-perform their robotic counterparts in tasks requiring dexterous manipulation. This is due to gaps in key technology areas, especially perception, robust high fidelity sensing, and planning and control. The roadmap for human-like dexterous manipulation consists of the following milestones:
\begin{itemize}
	\item 5 years: Low-complexity hands with small numbers of independent joints will be capable of robust whole-hand grasp acquisition.
	\item 10 years: Medium-complexity hands with ten or more independent joints and novel mechanisms and actuators will be capable of whole-hand grasp acquisition and limited dexterous manipulation.
	\item 15 years: High-complexity hands with tactile array densities, approaching that of humans and with superior dynamic performance, will be capable of robust whole-hand grasp acquisition and dexterous manipulation of objects found in manufacturing environments used by human workers.
\end{itemize}}
\end{quotation}
In this set of milestones, dexterous manipulation is not mentioned at all for the 5 year mark.   Dexterity is expected in limited form at 10 years, and we are expected to achieve dexterous manipulation in robot hands in 15 years time.

{\it We propose to enable considerable dexterity at the 5 year mark through task directed design with focus on joint limits and compliance.}

There are many very excellent research groups working on the critical areas sensing, planning, and control.   Our approach is orthogonal and complementary to these efforts.    We aim to reduce the load on sensing, planning and control by designing the mechanism -- the robot hand -- to be as favorable to the intended task set as possible, i.e., by developing new analysis and design tools for crafting the robot hand mechanism to make dexterous manipulation easier from the start.

Our approach centers around several elements:
\begin{itemize}
  \item {\bf Develop a "Grasp Net" benchmark focused on realistic everyday manipulation at its full complexity.}   We have observed from human studies that there are a relatively small number of grasping tasks and manipulation actions that are used over and over, and that common and useful grasp families are connected to one another with frequently used manipulation actions.   We propose to create grasp net benchmarks consisting of grasp families and manipulations to move between them.   Mechanisms will be designed specifically to accomplish these grasp nets.
	
\item {\bf Design to optimally accomplish specific collections of manipulation tasks.}   We propose to design hands from the ground up with the goal of performing a specific set of grasping and manipulation tasks robustly.    From the beginning, these sets of tasks will include families of objects for generality.   They will reflect real-world manipulations from the Grasp Net in their full complexity.   But they will be concrete lists of benchmarks that the hand must accomplish, facilitating rapid evaluation and design exploration.   The goal is to pursue generality while rooting our evaluation of suggested hands solidly in useful tasks in the real world.

\item {\bf Strategically Place Actuators, Joint Limits, and Compliance.}  We believe that the keys to robustness of hand designs will be selection and placement of actuators, generous use of joint limits, and strategic use of built-in compliance.    We will explore a variety of actuator designs and mechanisms to add limits and passive compliance.    Theoretical contributions to grasp analysis and optimization will be bolstered and improved with experimental data collected from many design iterations, component testing, and system testing. 

\end{itemize}	
	
PI Pollard brings decades of experience in grasping and manipulation analysis and experience working with various robotic hands and systems.   PI Coros brings experience in design of creative, complex mechanisms to accomplish user specified tasks, as well as 10 years of experience in control algorithms for complex tasks.     This combination of skills is critical for designing robot hands with the capability and skills to perform robustly exactly the interactions that are needed for scenarios where humans and robots live and work together.

