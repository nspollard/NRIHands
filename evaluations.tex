\section{Evaluations}

Each unit of research will of course be evaluated as we go, but we pose two interesting overall evaluations here.

\begin{itemize}

  \item Generality.   If we design for specific task families, will the resulting robot hand be robust and capable enough to generalize beyond the specific given examples?   Because we place such importance on robustness in the design process, we believe the answer will be yes.   We will test generality with leave one out tests and by extent of ability to extrapolate beyond the bounds of the object set given in the Grasp Net.

  \item Comparison to existing robot hands.    Is it possible for existing robot hands to accomplish all grasps and manipulations within the grasp net?   If not, what fraction can they achieve, based on kinematic structure and load capabilities alone?    We can obtain experimental comparisons of our new hands vs. the Shadow, Barrett, Robotiq, Kinova, and other Hands available to us at CMU.   Our hypothesis is that we will be able to exceed capability and robustness of existing hands in traversing the Grasp Net Benchmark with relatively few actuated degrees of freedom and low cost.   We anticipate that the result may look quite different from the typical dexterous hand existing today.

\end{itemize}