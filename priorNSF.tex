\section{Results from Prior NSF Support}

\paragraph{Pollard's prior work on physics-based grasping and manipulation.} 
PI Pollard has extensive experience in grasp and manipulation planning, transfer, and optimization.   She has developed algorithms for grasping and manipulation that allow fast transfer of examples from human to robot and generalization to varying object geometries in a manner that is both fast and provides bounds on performance \cite{pollard2004closure,Pollard:WAFR02,pollard2005physically,Li:graspDB07}.   She and her students have performed numerous human subjects studies to better understand the complex process of grasping in real-world situations \cite{Chang:2009,Chang:JMB10,illing2014changing,liu2016annotating} and followed up with more highly capable robot planning algorithms and quality metrics that function well for robot grasping in the presence of uncertainties \cite{Chang:ICRA10,Kappler:2012,kim2013physically}.   She has studied human-in-the-loop manipulation (teleManipulation) \cite{Toh:2012,chung2015quadratic,Kim:CGA11} and examined anatomical considerations of the hand that may lead to better robot hand designs \cite{pollard2002tendon,fu2006importance,Chang:twoAxis08,Chang:CoR06,Chang:AoR06}.


\paragraph{Results from Prior NSF Support.}
Pollard's NSF award most relevant to this proposal is \emph{CCF: Capturing and Animating the Human Hand: Robust Recovery of Hand-Object Interactions} NSF CCF0702443
(PI:  Pollard  6/07 - 5/11, \$325,000).
{\bf Intellectual Merit:}  Results include a taxonomy of manipulation actions prior to grasping, findings on rotation prior to grasping, a novel algorithm for planning robotic tasks with preparatory manipulation, robust
algorithms for capturing human hand skeletal structure, fast algorithms for simulation of deformable systems, a novel interactive system for guiding
simulations, a system for robot teleManipulation using multitouch, and an investigation of new
representations for motion that are meaningful across 
individuals and species (including robots).  This award resulted in the following
publications
\cite{Toh:2012,Chang:2014,Gatesy:2011,Kappler:2012,Kim:ToG11,Kim:CGA11,Koonjul:ICRA11,Chang:JMB10,Chang:ICRA10,Kappler:Humanoids10,Chang:2009,Chang:twoAxis08,Chang:Humanoids08}.
{\bf Broader Impact:}  In addition to top venues in Robotics and Computer Graphics, results
have been published in the Journal of Motor Behavior, the Journal of
Biomechanics, the Journal of Theoretical Biology, and
in the book ``The Human Hand: A Source of Inspiration for
Robotic Hands.''  This grant supported one female PhD student, Lillian Chang,
whose recent dissertation contained many of the core research findings
supported by this award.  It also provided partial support for two masters students, three undergraduates,
and one postdoc.


\paragraph{Coros's prior work on [LOTS OF AWESOME AND RELEVANT THINGS].} 

[STELIAN, CAN YOU DO SOMETHING LIKE MY PRIOR WORK PARAGRAPH ABOVE FOR STUFF RELEVANT TO THIS PROPOSAL]

\paragraph{Results from Prior NSF Support.}
PI Coros is an NSF beginning investigator.