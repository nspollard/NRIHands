\section{The Grasp Net Benchmark}   

\begin{figure}
\begin{center}
%{\includegraphics[width=6in]{./images/test.png}}
\vspace*{2in}
\end{center}
\caption[]{(Left) The 33 grasps of the Feix et al. taxonomy can be grouped into six classes based on how the hand is used. (Middle) Some examples of transitions between and within groups.  (Right) Marker set and sample data from capture of a preliminary Grasp Net.}
\label{GraspNet}
\end{figure}


In a series of human subject studies \cite{Liu2014, JiaDatabase, Chang:2009:RSSWorkshop, Chang:2014, liu2016annotating}, we have observed many commonalities between grasps and manipulations to acquire those grasps and move between them.   We show some examples here to motivate the idea that simple organizing principles exist.

Consider first the grasp taxonomy recently developed by Feix and colleagues \cite{feixgrasp}, which pulls together the wealth of research on grasp classification over the last century.   We find that the 33 grasps from this cumulative taxonomy can be placed into six groups (Figure \ref{GraspNet}, unpublished research by PI Pollard).  Variations within a group depend largely on object geometry.

Furthermore, we observe common transitions between these categories, such that a brief motion causes a robest transition from one grasp to another.   The Figure shows several examples.   The collection of grasps and transitions consists of various connected components.   One aim of this proposal is to fill in the gaps in our current Grasp Net through directed human subjects experiments.   We have already collected one dates containing 21 grasps and more than 25 manipulations, forming a preliminary real-world Grasp Net.   The Figure shows our marker set and a sampling of fingertip positions in the hand coordinate frame for this dataset.
