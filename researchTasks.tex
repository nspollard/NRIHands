\section{Research Tasks and Questions}
  \label{secQuestions}

Primary tasks and research questions for this project follow.

\smallskip\noindent
{\bf (1) Extend Quasistatic Grasp and Manipulation Analysis approaches to consider Compliance and Joint Limits.}  Extend the simple example of Section \ref{secLimitAnalysis} to more complex scenarios, including multiple degree of freedom fingers, redundant mechanisms, actuators and links of various types, different varieties of joint limit surfaces.   Our goal is to quickly evaluate the effect of any design change so that alternative designs can be explored and the hand can be optimized in a very fast search loop.

\smallskip\noindent
{\bf (2) Fast techniques for Evaluating Robustness.}   Quasistatic Grasp and Manipulation Analysis as discussed in (1) considers a specific scenario.   However, we care very much about robustness to uncertainty, even (especially!) in the mechanism design stage.  In Section \ref{secManipAnalysis}, we showed how robustness could be evaluated through rollouts of the manipulation under samples of a distribution of likely scenarios.   Using rollouts to evaluate success is functional, but slow.    Using simulation rollouts may not scale well to the breadth of situations we plan to address.

\begin{figure}
\begin{center}
%{\includegraphics[width=6in]{./images/wrenchPickup.png}}
\vspace*{2in}
\end{center}
\caption[]{[ADD ALON PIC AND ROBOT EXAMPLES.  (Left) Our manipulation task can succeed for any of these (and more) geometries.   Inclusion in this set can be determined by linear projection.   (Right) Two examples of performance of the same manipulation on different geometries.   We propose to develop similar fast methods to test robustness of a mechanism to perform a manipulation in the inner loop of a design process.}
\label{AlonManip}
\end{figure}


Previous research by PI Pollard has resulted in techniques that can use a simple and fast linear projection to analyze robustness to errors in placing the fingers on an object \cite{pollard2004closure}.   We have extended the process to evaluate robustness for manipulation tasks \cite{Pollard:WAFR02} (Figure \ref{AlonManip}) and shown that the same approach can be used to express ability to grasp objects of different geometries \cite{pollard20045} and to work with tendon driven systems \cite{Li:graspDB07}.

Briefly, the idea we have exploited in this previous research is to portion variations out to different parts of the mechanism (e.g., to each contact).   As long as each contact is confronted with a situation within its portioned space of variations, we can guarantee that overall, the task can be completed at least X percent as efficiently as an example.   Decreasing X gives more freedom, but less efficient results (e.g., for lower X we may need to reduce weight limits on manipulated objects).   Unlike many other approaches, which scale exponentially with number of contacts, this approach scales only linearly with number of contacts and works better and better as number of contacts increases; with large numbers of contacts comes great ability to adapt to uncertainties and variations.   Many human manipulation tasks benefit from large numbers of contacts and we expect to exploit this property.

Use of this idea in a design process is as follows.   Suppose a mechanism design for which we wish to test robustness.    Given a  solution to manipulate one object, we develop a test that utilizes a small number of linear projections to evaluate the expected success of any combination of uncertainties in mechanism and variations in object geometry, following the approach of apportioning out variation spaces to different parts of the mechanism as outlined above.   These projections will be trivially fast and will not require analyzing details of each object or computing a mechanism trajectory.

Funding for this proposal will give us the opportunity to investigate this idea properly.   The potential impact is large, both for manipulation planning and mechanism design, as having linear projections in the inner loop of a planning or optimization process as opposed to simulation rollouts to evaluate the same thing can speed these operations by orders of magnitude, allowing real-time planning and user-in-the-loop interactive design optimization.

\smallskip\noindent
{\bf (3) Design Optimization Tools that Scale to Large Grasp Nets and manipulator complexity.}   [STELIAN FLESH THIS OUT A BIT]

 \smallskip\noindent
{\bf (4) Understanding the Delta Provided by different Sensing Approaches.}   Sensor design will be considered in later stages of the project with the point of view of what type of sensor could most improve performance of the mechanism on its intended tasks (e.g., increase generality or robustness and/or eliminate catastrophic failure).   We hypothesize that the best role for sensing in many manipulation actions is to identify when things are going very wrong and to make a straightforward adjustment to compensate.   Reflexes are a good example, response to slipping of the grip, total loss of contact, or stopping of expected motion.   One type of response may be to  lift the object and move it aside to clear an obstacle.     It is encouraging that in our human subjects studies we observe frequent failures such as collisions prior to reaching the intended destination, which are resolved (after a delay of approximately 100ms) with characteristic corrections.    Our first line of attack will be to enable similar behavior.  The proposed research  will assess sensors as any other design element in terms of their expected value in completing the intended manipulation tasks.

     

