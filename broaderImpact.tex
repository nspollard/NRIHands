
\section{Broader Impact}

The expected result of this proposal is robot hands that are robust, inexpensive, dexterous, and as simple as they can be to achieve their function.     The long term promise is dexterous co-robots that can contribute in all corners of life, rendering competent assistance in manual tasks to improve our quality of life, our health, and our safety.    Example application areas include manufacturing, rehabilitation and prosthetics, assembly / disassembly operations in space, and personal robots that are able to assist with shopping, cooking, maintenance, dressing, and feeding, allowing elderly or disabled individuals the dignity to live longer in their own homes.

\paragraph{International Collaboration.}
We are have active collaborations with ETH Zurich, we host students from the University of Karlsruhe in Germany, and we anticipate productive collaborations with Paul Kry's group at McGill University in Canada in connection with the proposed research (see letter).    

 \paragraph{Graduate education.} This project will have a significant impact on graduate students.   Students will be encouraged to work on the project directly, and to experience the research through course offerings at CMU, including 
 15869 Computational Aspects of Fabrication and 16899 Hands: Design and Control for Dexterous Manipulation.  

 \paragraph{Undergraduate education.}   We expect numerous undergraduates to be involved in this research as well.    PI Pollard has advised more than 20 undergraduates, more than one third female, most involved in NSF supported research, and many who have gone on to graduate school.    Both PIs offer advanced undergraduate courses that will expose undergraduates to the proposed research and encourage them to get involved themselves.

 \paragraph{Participation of underrepresented groups.}  Pollard has mentored four women through the CRA-W Distributed Research Experiences for Undergraduates (DREU) program, and has advised 7 female graduate students and 8 female undergraduates, three of whom have gone on to PhD studies at Stanford and CMU.

 \paragraph{Public and K-12 outreach}
Pollard has been involved in discussing robotics and 3D printing of functional objects at local elementary schools to increase interest in STEM and Maker topics, especially with young girls.  The proposed research offers a great opportunity for K-12 outreach, due to the hands-on nature of the robot hand prototypes which will be generated at many stages of the research project.   We will introduce young students to these prototypes and to robotics in general, both through annual CMU outreach programs and through our own school visits and talks.

 \paragraph{Dissemination.} 
 Dissemination will be primarily through publication in conferences and journals and participation in and organization or relevant workshops.    Prototype designs will be made available in the form of design files and instructions to speed progress in this critical area.   Human manipulation data collected as part of this project will be made available online.   Grasp Net manipulation benchmarks will be provided as metrics for testing potential and actual hand designs for dexterity, contributing to the valuable efforts that have resulted in the NIST ``Roadmap to Progress Measurement Science in Robot Dexterity and Manipulation" \cite{falco2014roadmap} and the YCB Object and Model Set benchmarks for manipulation research \cite{calli2015ycb}.

