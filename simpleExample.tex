\section{A Simple Example}

In this section, we work out a simple example for purposes of illustrating our proposed approach.

\begin{figure}
\begin{center}
%{\includegraphics[width=6in]{./images/wrenchPickup.png}}
\vspace*{2in}
\end{center}
\caption[]{This trivial Grasp Net nonetheless captures two important grasps and a dexterous motion to move between them.}
\label{SimpleGraspNet}
\end{figure}

Consider the trivial Grasp Net show in Figure~\ref{SimpleGraspNet}.     This figure shows two grasps.   Grasp A is a pinch grasp, typical for lifting objects from a surface, placing them down, performing certain dexterous actions such as using tweezers, and as a staging point for moving to other grasps (e.g., consider lifting a wrench and then maneuvering it to enclose it within the palm).    Forces in the pinch grasp oppose one another along the local y-axis.   Objects may be of varying width, and grasp force may vary, creating a family of pinch grasps we would like the robot to be able to achieve.

Grasp B is a lateral grasp, often called the key pinch grasp, useful for comfortably and securely holding objects, for certain assembly operations (e.g., put a key or a card into a slot), and for offering an object to a person (or another robot).   Forces for the lateral grasp in this example oppose one another along the x-axis.  We consider the same object set, but rotated 90 degrees.   Greater forces may be desired for this grasp.   As with all grasps, we specify an exact set of relative contact locations and forces that defines our grasp.

Manipulation 1 is used to move between the two grasps.   In this case we can imagine the "thumb" pivoting around the "index finger," although in our final design, the motion may be generated by either or both fingers.    The variation in object widths creates a family of curves describing motion of the thumb relative to the finger.   The object is specified to slide on the finger so that the thumb contact remains nearly constant.

Forces in this example do little more than guarantee that the object remains secured while performing the manipulation.   The manipulation is assumed to be quasistatic.

A successful design must be able to move all of the benchmark objects from Grasp A to Grasp B using Manipulation 1.

Suppose that a mechanism designer has specified the following design constraints and suggestions:
\begin{itemize}
   \item Actuators apply unidirectional linear force in the finger coordinate frame (e.g., tendons acting through the finger center of mass).
   \item Fingers can be made to be passively compliant using a linear stiffness model.  [STELIAN GIVE ME A BETTER STIFFNESS MODEL.]
   \item At any time instant, at least one force must be active.
   \item Force directions during static grasping may be good directions in which to actuate a finger.
   \item Force directions during static grasping may be good directions in which to craft joint limits.
\end{itemize}

Suppose further that the designer's goals, in order of priority, are:
\begin{enumerate}
	\item Grasp and manipulate all benchmark objects as specified in the Grasp Net.
	\item Minimize the total number of actuators.
	\item Minimize the number of actuators per finger.
	\item Minimize the sum of forces that are actively applied (i.e., joint limits and passive forces due to compliance are "free")
\end{enumerate}

For Goal 2, a test of rank on relative positions will determine that the minimum number of actuators is two.

For Goal 3, a tree search will find we can split these actuators between the fingers, so that each has one degree of freedom.

Now, we are left with 8 choices, as the thumb can be actuated in any of the four coordinate axis directions, and in each case there would be two choices of direction in which to actuate the index finger (specifically [x, y], [x, -y], [-x, y], [-x, -y], [y, x], [y, -x], [-y, x], [-y, -x]).

Figure~\ref{SimpleExampleResults} shows one final solution, where the thumb is actuated along the negative x-axis and the index finger is actuated along the negative y-axis.   This is perhaps the most humanlike example and most resembles hands which actuate the grasping forces and use return springs to bring the fingers back to a fully open pose.   

[SAY SOMETHING ABOUT JOINT LIMITS AND SPRINGS.]

[EXPLORATION:   (1) THUMB GETS Y .. ANY DIFFERENCE?   (2) VALUE OF 4 ACTUATORS?]

We can compare work done in the various cases, including without compliance.

\begin{figure}
\begin{center}
%{\includegraphics[width=6in]{./images/wrenchPickup.png}}
\vspace*{2in}
\end{center}
\caption[]{Final solution.}
\label{SimpleExampleResults}
\end{figure}


