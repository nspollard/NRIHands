\section{A Simple Example}


\begin{figure}
\begin{center}
%{\includegraphics[width=6in]{./images/wrenchPickup.png}}
\vspace*{2in}
\end{center}
\caption[]{This trivial Grasp Net nonetheless captures two important grasps and a dexterous motion to move between them.}
\label{SimpleGraspNet}
\end{figure}

Consider the trivial Grasp Net shown in Figure~\ref{SimpleGraspNet}.     This figure shows two grasps.   Grasp A is a pinch grasp, typical for lifting objects from a surface, placing them down, performing certain dexterous actions such as using tweezers, and as a staging point for moving to other grasps.   We observe that many objects are initially acquired from a surface in a pinch grasp and then moved to a final grasp that is more useful, comfortable, or powerful.    Forces in the pinch grasp oppose one another along the local y-axis.   

Grasp B is a lateral grasp, often called the key pinch grasp, useful for comfortably and securely holding objects, for certain assembly operations (e.g., put a key or a card into a slot), and for offering an object to a person (or another robot).   Forces for the lateral grasp in this example oppose one another along the x-axis.  Greater forces may be desired for this grasp.   

Manipulation 1 is used to move between the two grasps.   In this case we can imagine the "thumb" pivoting around the "index finger," although in our final design, the motion may be generated by either or both fingers.    The variation in object widths creates a family of curves describing motion of the thumb relative to the finger.   The manipulation is specified to utilize sliding contact on the finger and rolling frictional contact with the thumb, such that the point of contact between object and thumb remains nearly constant.  Forces are specified in this example to do little more than guarantee that the object remains secured while performing the manipulation;  the manipulation is assumed to be quasistatic.  

There are N benchmark objects of varying width and required grasp force.   The design problem is to choose degrees of freedom, place actuators and joint limits, and select passive compliance for the mechanism.  A successful design must be able to move all of the benchmark objects from Grasp A to Grasp B using Manipulation 1.

Suppose that a mechanism designer has specified the following design constraints and suggestions for this toy example:
\begin{itemize}
   \item Actuators will create linear motion in the x-y plane.
   \item Fingers are passively compliant using a linear stiffness model.
   \item At any time instant, at least one force must be active.
   \item Force directions during static grasping may be good directions in which to actuate a finger.
   \item Force directions during static grasping may be good directions in which to craft joint limits.
\end{itemize}
Suppose further that the designer's goals, in order of priority, are:
\begin{enumerate}
	\item Grasp and manipulate all benchmark objects as specified in the Grasp Net.
	\item Minimize the total number of actuators.
	\item Minimize the number of actuators per finger.
	\item Minimize the sum of forces that are actively applied (i.e., joint limits and passive forces due to compliance are "free")
\end{enumerate}

For Goal 2, a test of rank on relative positions will suggest a need for two actuators.  Goal 3 will split these actuators between the fingers, so that each has one degree of freedom.  Figure~\ref{SimpleExampleResults} shows one final solution.  [EXPLAIN]

[TODO:  INVESTIGATIONS]
\begin{itemize}
   \item Minimize work.   Without compliance.   With compliance.   With four instead of two actuators.   Leave one out tests.   Goals:   show compliance matters, show some generality can be obtained.
   \item Maximize robustness.   Allow shape changes.   Test with and without joint limits.   Goals:   show joint limits and compliance both affect robustness.
   \item Simplify control.   Optimize for robust performance with a simple control motion.   Show that if we consider control in the design, it can be made very simple and we don't sacrifice robustness.   Can we get away with a single motor and drive the system in both directions?
\end{itemize}

\begin{figure}
\begin{center}
%{\includegraphics[width=6in]{./images/wrenchPickup.png}}
\vspace*{2in}
\end{center}
\caption[]{One final solution.}
\label{SimpleExampleResults}
\end{figure}


