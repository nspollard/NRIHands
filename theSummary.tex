\section{Summary}

Dexterity is a Grand Challenge goal in robotics today, and is on the critical path to capable robots in almost every domain:  home, work, space, medicine, disaster scenarios, any situation where a robot must interact to effect change in the world.    With advances in rapid prototyping and design optimization, we would appear to be well poised for dramatic progress in robot dexterity.   Rapid prototyping technologies are becoming available in every lab.    New robot hands are being designed at an unprecedented pace.

However, we are missing one crucial ingredient:  we do not yet fully understand manipulation.   We have no clear guidelines for creating the robot dexterity that would cause a transformative advance.   The lack of clear design goals is a great barrier to further progress, and we as a field are struggling greatly to create reliable, apparently effortless dexterous behavior.

Our proposal aims to address this gap.   We have observed patterns in human manipulation strategies and actions such that we are able to create Grasp Nets representing families of dexterous behavior.    These Grasp Nets provide the design focus that has been lacking.    Our expertise in grasp and manipulation analysis (PI Pollard) and optimal design, rapid prototyping, and control (PI Coros) make us the perfect team to convert these observations  into robot hands that present previously unseen levels of robust dexterous behavior.

{\bf Intellectual Merit:}
Our working principles and research questions are that:
(A) On-task compliance reduces actuator requirements.   This is important for dexterous manipulation, because keeping number of actuators low and size of actuators small makes it possible to design lightweight hands with less tendon routing or other actuator complexity.
(B) Joint limits improve robustness to error and likely make it easier to learn a task.
(C) Working from specific Grasp Nets (such as we observe people to use) focuses design efforts to make possible the next leap in design for dexterous manipulation.

We will advance the state of the art in:
(1) Grasp Nets capturing manipulation capability observed in human performance.
(2) Novel optimization techniques to design mechanisms tuned for manipulation.
(3) Mechanism optimization considering joint limits and compliance.
(4) Analysis of manipulation with joint limits.
(5) Fast tests for robustness through linear projection of uncertainties into geometric subspaces.
Together, these contributions will make possible for the first time reliable and broadly varying dexterous behavior from accessible prototyping technologies.

{\bf Broader Impact:}
Having dexterous robots will affect many areas of our lives.   Dexterous robots can render competent assistance in manual tasks to improve our quality of life, our health, and our safety.     Educational broader impact includes K-12 outreach efforts.  For example, we have recently brought the discussion of robot hands in to the elementary school classroom.    Evaluation metrics will be available as grasping and manipulation benchmarks.   Grasp Nets and robot hand designs will be made available so that others can access and build upon our results.   The PIs have an excellent track record with mentoring underrepresented students and undergraduates and will continue this trend.


