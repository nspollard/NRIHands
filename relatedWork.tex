\section{Comparison to Related Work}

Researchers have thought a lot about how to evaluate grasps, plan grasps

but design of a mechanism for a specific suite of grasp and manipulation actions that is meant to be general is less well considered.

Ciocarlie
Frank Hammond III

Trinkle lever-up
long train of Bicchi analysis papers


Many dexterous robot hand designs attempt to be as capable as possible -- e.g., max manipulability.  

In a way, we take the opposite approach -- as capable as it needs to be and no more.   Because limits help us.

Traditional grasp and manipulation analysis assumes we operate the hand far from singularities and joint limits.	   In such a case, the dynamics of the system may be linearized and used to estimate our ability to control an object and apply and resist forces within a local configuration space [refs].   Such estimates may be used as a basis for hand design [ref].

However, such analyses are limited when it comes to understanding the broad spectrum of capabilities of a proposed robot hand, and designers have formed other tests, such as workspace analyses [Feix comparing to human hand] and heuristics such as ability to oppose the thumb to each of the fingers.

Abeel results suggest that learning the manipulations on the real robot is possible with small numbers of iterations.   We expect even faster learning and more generality because the mechanism is designed exactly to do this.