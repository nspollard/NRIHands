\section{Comparison to Related Work}

Design optimization for robot hands has been considered by a number of research groups, and just a few examples are mentioned here.   Salisbury  designed the Stanford/JPL hand to optimize ability to manipulate a small object held in the fingertips \cite{salisbury1982articulated}.  Dollar and his colleagues optimize hands to improve ability to capture the object in a grasp, to maximize workspace while holding a grasped object, and to consider the manipulation task of lifting an object from a table with a two actuator hand \cite{borrasDollar2015,ma2014linkage,odhner2015stable}.     Ciocarlie and colleagues optimize a hand to grasp objects securely, whether they be small and flat or larger and round by keeping the distal links parallel until contact between hand and object\cite{ciocarlie2014velo}.    Hammond III and colleagues consider which joint ranges of motion can be reduced (or even eliminated) while maintaining ability to achieve a fixed set of grasps \cite{hammond2012towards}.   Bicchi and his colleagues have explored the idea of adaptive synergies in the design of robot hands \cite{catalano2014adaptive}.   Birglen and Gosselin discuss optimization of underactuated hands in part to avoid the phenomenon of ejection \cite{birglen2006geometric}.    We are inspired by these and other efforts in robot hand design optimization and wish to take the next leap to optimize for complex manipulation tasks observed to be critical for dexterity in human environments.

There have been many exciting highly dexterous hands designed, some with the explicit goal of duplicating or exceeding the capabilities of the human hand.   Just a few examples are found in these references \cite{jacobsen1986design,vande2004act,ShadowHand,mouri2002anthropomorphic,lovchik1999robonaut,grebenstein2011dlr,xudesign}.    Perhaps in contrast with some of these works, we begin with the goal to achieve a given network of grasps and manipulation actions, with the approach to make the hand exactly as capable as it needs to be -- not less and perhaps not more -- because adding limits can help.   Our hand may or may not look much like the human hand.   How it comes out will depend on the design elements we provide and what is needed for it to do the required manipulations in a robust manner.

There has been a virtual explosion of ideas on how to use rapid prototyping techniques in robot design \cite{dollar2010highly}, in soft robot design \cite{hirose1978development,deimel2015novel,rus2015design,wehner2014pneumatic,polygerinos2015modeling}, and design of compliant elements such as joints \cite{kuo2016novel}, making this a great time to be exploring the research problems of this proposal.

There has been much work on grasp and manipulation analysis that is also very inspiring \cite{trinkle1990planning,prattichizzo2013motion,prattichizzo1997consistent,lin2000stiffness}.  We especially appreciate the points of view that we can use ``defective" mechanisms to our advantage, that local stiffness properties are important for computing quality metrics, and that we wish to configure the mechanism so that abstractly ``if we just squeeze," good things will happen.    We expect to build on and extend this research, which will become more and more valuable as we build robots which exploit limits, singularities, asymmetries, and highly coupled action as a matter of course.

The research of Abeel with Levine and colleagues \cite{finn2015deep,kumaroptimal,levine2015learning} suggest that learning manipulations on real robots is possible with small numbers of iterations.   We expect even faster learning and more generality and robustness because the mechanism is designed exactly to do the manipulations it is performing.