\documentclass[11pt]{article}

\renewcommand{\topfraction}{.75}
\setcounter{topnumber}{10}
\renewcommand{\textfraction}{0}
\renewcommand{\thesection}{\Alph{section}.}
\renewcommand{\thesubsection}{\arabic{subsection}}
\renewcommand{\thesubsubsection}{\arabic{subsection}.\arabic{subsubsection}}
%\renewcommand{\thepage}{\Alph{section}-\arabic{page}}
\renewcommand{\thepage}{\Alph{section}-\arabic{page}}

\addtolength{\textheight}{1.40in}
\addtolength{\textwidth}{2.0in}
\addtolength{\oddsidemargin}{-1.2in}
\addtolength{\topmargin}{-0.2in}
\baselineskip = 0.9\baselineskip


\setcounter{section}{5}

%\def\item{\hangindent 24pt $\bullet$}
\def\descriptionlabel#1{\hspace\labelsep #1}
\long\def\COMMENT#1{}

%\def\item{\hangindent 24pt $\bullet$}
\def\descriptionlabel#1{\hspace\labelsep #1}

\usepackage{times}

\begin{document} 

\begin{center}
{\large \bf Facilities, Equipment, and Other Resources}
\end{center}
 
\noindent
The CMU Motion Capture Lab has a large space devoted to motion capture and robotic experiments. Facilities include a working kitchen and a selection of robots which can be setup in the capture environment. We maintain a full time highly experienced staff.  Our motion capture system features 16 state of the art Vicon cameras capable of capturing detailed motion of hundreds of markers at 120 frames per second. We have available a cyberglove for capturing hand motions and pressure sensing glove for capturing forces.    We maintain a 15 DoF Motoman SDA10 torso and two arm robot, a 24 DoF Shadow compliant (pneumatic) robot hand, and multi-touch telemanipulation system.   


\paragraph{Heterogeneous Distributed Computing}
The Carnegie Mellon University (CMU) School of Computer Science (SCS) research facility has a large number and wide
variety of computers available for faculty and graduate student use approximately 4000 machines. About one-third
are Linux/Unix on Intel and AMD platforms, 60\% are Windows systems, and Macintosh computers make up the
remainder. Every incoming graduate student is provided with a new, high-powered personal computer; some receive a
dual-boot configuration - both Windows and Linux. SCS facilities include a rich variety of computing infrastructure
services of very high quality: email, shared file service (AFS), authentication, remote access services (VPN, iPass),
backup, printing, software licensing, hardware repair, and so on. The SCS environment also includes a growing number
of high-performance compute clusters; support services are available for the entire life-cycle of the cluster, including
help for specification and purchasing. For all aspects of computing, there is a dedicated support (Help) staff within the
facility, which provides full support for users, applications, machines, and services, via a menu of premium support
services.
Beyond these college resources, the University maintains computation facilities of various kinds for general use. The
Pittsburgh Supercomputing Center (PSC) is a joint effort of Carnegie Mellon and the University of Pittsburgh together
with Westinghouse Corporation. It is supported by several federal agencies, the Commonwealth of Pennsylvania and
private industry. It is a leading partner in the TeraGrid, the National Science Foundation's cyberinfrastructure program.
It operates several supercomputing-class machines, including an SGI UV shared memory machine with 4096 cores and
32TBytes of shared memory.

\paragraph{Networking}
Carnegie Mellon operates a fully-interconnected, multimedia, multiprotocol campus network. The system incorporates
state-of-the-art commercial technology and spans all campus buildings in a redundant 10Gbps backbone infrastructure
that enables access to all campus systems, including the PSC supercomputers. The University also provides Wi-Fi
connectivity in all campus buildings; it has recently upgraded to equipment that supports the 802.11n standard, which
provides wireless speeds in excess of 100Mbps.
SCS has redundant 10Gbps links to the Carnegie Mellon campus network. The University has redundant 1Gbps links to a
combination of providers for internet connectivity. These include Sprint and Level3 for commodity internet traffic, and
the Three Rivers Optical Exchange (3ROX) for connections to a number of high-speed research and education networks,
including Internet2, National Lambda Rail, ESnet, and Teragrid. The University can also arrange advanced point-to-point
research connectivity through services such as Internet2's Dynamic Circuit Network.

\paragraph{General Facilities Information}
Carnegie Mellon's School of Computer Science is the largest academic organization devoted to the study of computers.
Its seven degree-granting departments --- the Computer Science Department, the Human-Computer Interaction Institute,
the Institute for Software Research, the Lane Center for Computational Biology, the Language Technologies Institute,
the Machine Learning Department, and the Robotics Institute --- include over 250 faculty, 700 graduate students, and a
250-member professional technical staff. SCS also collaborates with other University Research Centers, including the
Software Engineering Institute (SEI), the Pittsburgh Supercomputing Center (PSC), the Information Networking Institute
(INI), the Institute for Complex Engineered Systems (ICES), the Center for the Neural Basis of Cognition (CNBC), and the
Entertainment Technology Center (ETC).

\paragraph{Digital Fabrication Equipment}
PIs Pollard and Coros have access to a state of the art, semi-professional Fused Filament Fabrication 3D Printer (Stratasys UPrint SE). 
This 3D Printer will enable the fabrication of the robotic hand prototypes developed as part of our proposal.


\vfill

\end{document}
