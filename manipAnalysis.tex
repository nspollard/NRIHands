\section{Dexterous Manipulation Analysis with Joint Limits and Compliance}

\begin{figure}
\begin{center}
%{\includegraphics[width=6in]{./images/wrenchPickup.png}}
\vspace*{2in}
\end{center}
\caption[]{Several designs with the goal of robust pushing.}
\label{PushExample}
\end{figure}

\begin{figure}
\begin{center}
\begin{tabular}{l|c|c|c|c|c|c|c|}
Design & x0 & y0 & kx & ky & cost &	actuated x range &	actuated y range \\ \hline
(A) 4 tendons, no compliance&	-&	-&	-&	-&	1600	&(-4, 0)	&(-4, 0)\\
(B) 4 tendons, compliance	&-1.887	&-2.58	&0.54&	0.43&	{\bf 107.65} &	(-1.36, 1.02)	&(-1.61, 1.11) \\
(C) 2 tendons, spring open&	7	&7	&min	&min	&1600	&(-4, 0)	&(-4, 0)\\
(D) 2 tendons, spring closed&	-5.883&	-4.998&	0.45&	0.5	& {\bf 445.3}  &	(0, 2.65)&	(0, 2.49)\\
\end{tabular}
\end{center}
\caption{Optimal parameters, cost, and actuator range for several mechanism designs to perform Grasp A, Grasp B, and Manipulation 1.}
\label{ComplianceAnalysis}
\end{figure}

From the simple example of Figures~\ref{SimpleGraspNet} and \ref{SimpleExampleResults}, we can show two things:  (1) Careful design of compliance can reduce the number of actuators and actuator load, and (2) Joint limits can be important for robust performance of a manipulation and perhaps also contribute to making the manipulation easy to learn.   Furthermore, using a different example shown in Figure~\ref{PushExample}, we show a little of how (3) we can expand our analysis tools to consider design with joint limits.  Finally, in the next section, we discuss optimization of complex mechanisms to accomplish Grasp Net tasks.

\smallskip\noindent
{\bf (1) Compliance matters.}  Table~\ref{ComplianceAnalysis} shows results for four tendon driven designs, which were optimized to perform Grasp A, Grasp B, and Manipulation 1 for 4 objects of varying widths.   Quasistatic analysis was performed for each design, with actuators for the thumb moving in the X direction, and actuators for the finger moving in the Y direction.  The quasi static analysis ensured that forces required for Grasp A, Grasp B, and Manipulation 1 could be generated by the mechanism.   Each design was optimized using CMA \cite{hansen2006cma} to require the minimum total actuation force, expressed as squared actuated forces summed over objects, actuators, and time samples.    Where compliance was considered, linear springs were assumed, and the algorithm optimized for spring zero point and stiffness.   Optimal parameters, cost, and range of actuated forces are shown in the Figure.

Design (A) contains 4 tendons, which give active control in both the negative and positive X directions for the thumb and the negative and positive Y directions for the finger.   This design considers no passive compliance.   Design (B) contains the same 4 tendons, but it also includes linear springs.   Note that the cost of forces provided by the actuators is less than 7 percent that of Design (A).    Design (C) has only two tendons, which are used as elastic return springs, opening the hand.    The presence of the spring elements reduces the total number of actuators, but does not reduce the cost of the design.    Design (D) also has two tendons, which are employed to open the fingers.    Here, the total cost is 28 percent of that of Design (A), and has been achieved with half the number of actuators.    Clearly, compliance matters to reduce the total number of actuators and to reduced actively applied forces for actions in the Grasp Net.

\begin{figure}
\begin{center}
\begin{tabular}{l|c|c|c|c|c|}
Test Case & \multicolumn{5}{c}{Errors at Gaussian perturbation (radians)}  \\
                & None & 0.1  & 0.2  & 0.3  &  0.4  \\ \hline 
(A) Joint limits, learn with perfect mechanism &	{\bf 0.01} & 0.90 & 1.63 &	2.32 & 3.61 \\
(B) Joint limits, learn with perturbation	&  {\bf 0.04}	& {\bf 0.05}	& {\bf 0.08} &	{\bf 0.21} &	1.42 \\
(C) No joint limits, learn with perfect mechanism &	{\bf 0.01}	&	1.05&	2.13	& 2.84  &	4.06 \\
(D) No joint limits, learn with perturbation &	{\bf 0.03}	& {\bf 0.06} 	&1.43	&4.38	&5.75 \\
\end{tabular}
\end{center}
\caption{This exploration shows the value of joint limits.   Mean error of an optimized manipulation is shown as perturbations to the perfect mechanism grow in magnitude.    Standard deviation is in parentheses.  The test having both joint limits and optimized while seeing small perturbations is able to handle perturbations three time the size of any other situation tested.}
\label{JointLimitAnalysis}
\end{figure}

\smallskip\noindent
{\bf (2) Joint Limits matter.}   We observe that joint limits are not avoided in human motion.   In contrast, in the human system, joint limits and singularities are often exploited.    The limit at our knee that facilitates straight leg walking is a well known example, but there are many examples in grasping and manipulation as well.  In a lateral grasp (which motivated Grasp B), the moderately flexed fingers can passively support large lateral forces, even though the force that cana be actively generated by forces in that grasp is small, coming from small intrinsic muscles in the hand, such as the interossei \cite{brand1999clinical}.   These large forces are available because the fingers are operating near their joint limits in this direction.   As another example, large forces can be transmitted through our palm all the way up to our strong shoulder muscles, or even through to the ground in some cases.   Pressing hard with a fingertip will also exploit joint limits.

Here, we show one simple example that joint limits can make a practical difference.   In this set of tests, we explored the ability to learn robust open loop control for Manipulation 1.    Motor velocities for the thumb and finger were represented as linear functions of time, with three velocity control points per finger, resulting in a search space of six parameters.   CMA was used to optimize these parameters.    Box2D was used to simulate the manipulation for any given parameter set.   The cost at the end of each simulation was the difference between the final object configuration and that desired for Grasp B.     Once an optimal open loop motion had been identified by the optimizer, it was evaluated on a series of 5 tests, each involving 100 simulations having perturbations in the direction of motion of the finger.    Finger motion direction was perturbed about its intended direction using a normal distribution with standard deviation of 0, 0.1, 0.2, 0.3, and 0.4 radians.    The motivation for this choice is that in a more complex hand, it is often difficult to know exactly where the fingers are relative to one another, and this variation can affect ability to manipulate robustly.   When a test used joint limits, the limits for the thumb were placed in the X direction, both at the thumb's origin in Grasp A and at a wide open position that gave clearance twice the width of the largest object.   Joint limits for the finger, when used, were placed in the Y direction, both at its destination in Grasp B and in a wide open position that gave clearance twice the width of the largest object.

Four cases were considered.   In Case (A), joint limits were included, and the manipulation action was optimized assuming a perfect mechanism.   Case (B) is similar to Case (A), but included random perturbations during the learning phase, having a standard deviation of 0.2 radians.    Cases (C) and (D) are variations where joint limits are not included.   Table~\ref{JointLimitAnalysis} shows the results.   Empirically, results with error less than 0.5 in our scale produced consistently good results.   These results show that when a task is learned assuming a perfect mechanism, the effect of joint limits is not clear.    However, when the learning process has the opportunity to see the perturbations, the optimizer learns to use the joint limits to create compensating strategies.    Case (B), including both joint limits and learning with perturbations, is able to handle three times the amount of disturbance of any other case.      Based on qualitative observation only, the cases having joint limits also appeared to present easier learning problems.

\smallskip\noindent
{\bf (3) We must expand our analysis tools to consider design with joint limits.}     Consider now the example in Figure~\ref{PushExample} A.     The goal here is to push into a surface, holding the object in more or less the given grasp.   There will be variation in the task, as the object may contact the surface at different points and with different surface contact normals, resulting in varying force directions and moment about the object center of mass.    Each finger has one tendon to actuate it, and the directions of actuation are shown.    We assume the two fingers on the sides have considerable compliance in the horizontal direction.   Actuating these fingers does not help either, as this will only cause the fingers to slide along the object surface.    With the design as shown in Figure~\ref{PushExample} A, the grasp will only be able to apply a very small range of task related forces to the object.

We can examine the force balance equations for this situation.   Because the fingers are very compliant, forces on the fingers must be balanced to prevent them from moving:
\begin{equation}
    f_{s, i} +     f_{a, i} +     f_{JL, i} +     f_{c, i} = 0
\end{equation}
Parameter $f_{s, i}$ is the spring force, with a component due to deviation of the initial contact location $c_i$ from rest position $c_{i, 0}$ and a second component due to object motion that results from unequalized forces $\Delta X_o$, which is intended to be small for a good grasp.   Matrix $K_i$ encodes stiffnesses and $G_i$ is the grasp matrix for finger i.
\begin{equation}
 f_{s, i} = -K_i ( c_i - c_{i, 0}) - K_i G^T_i \Delta X_o
\end{equation}
Parameter $f_{a, i} = D_i P_i A_i$ is force generated by the actuators as in \cite{Li:graspDB07}, and is expressed using activation levels $A_i$, such that $0 \leq a_i \leq 1$, diagonal matrix $P_i$ of maximum actuator forces, and matrix $D_i$ of actuator directions.
Parameter $f_{JL, i}$ is the novel component compared to other treatments that consider grasp quality measures for compliant systems (e.g., \cite{lin2000stiffness}), and can be expressed for a frictionless joint limit surface as:
\begin{equation}
	f_{JL, i} = -k_{JL} N_i S_i N^T_i      G^T_i \Delta X_o
\end{equation}
with joint limit normal matrix $N_i$ and (large) stiffness experienced at the limit of $k_{JL}$.    Note the selector matrix $S_i$ which indicates which joint limits relevant to finger i are currently engaged by some external or active force.
Parameter $f_{c, i}$ is the contact force, which must be within the friction cone at the contact and contributes to balancing a desired external wrench $W_{ext}$ due to the pushing contact with the ground in this case:
\begin{equation}
  - \sum_i G_i f_{c, i} = W_{ext}
\end{equation}
For a given value of selector matrices $S_i$, we can solve the resulting linear system for actuator forces $A$ and object motion $\Delta X_o$.    In a valid solution, $\Delta X_o$ must be consistent with $S_i$ such that the object motion pushes the object into the limit to engage it.   The system as a whole can be resolved through pivoting or other discrete search techniques.

This setup can also be used for design.   As mentioned above, the side fingers in Figure~\ref{PushExample} A are not at all helpful for accomplishing the task.  A search for optimal joint limits may result in Figure~\ref{PushExample} B, which is similar to using two fingers laterally as guides while pushing with a middle finger.   Here, the range of task forces that can be applied is considerably greater.     Torques due to ground contact forces at the object base engage one of the joint limits, allowing a force balance to be created.

A third option, shown in Figure~\ref{PushExample} C, is to add an additional actuator that can actively engage the joint limit opposite to it.   The result is a very strong grasp, reminiscent of human power lateral grasps, which can easily provide the task forces and many more.   As a side benefit, the two previously useless actuators can now contribute to the task, sharing the load with the actuator of the top finger.

In each case, the force balance equation as shown here, along with a straightforward linear optimization to compute the set of task forces that can be equalized (following our own work \cite{Li:graspDB07}), have made it possible to evaluate the effect of any new design element on the push grasp.

Extension to complex and redundant mechanisms is more complex, and is the subject of future research.

